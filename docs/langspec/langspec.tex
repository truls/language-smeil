\documentclass{article}

\usepackage[T1]{fontenc}
\usepackage{syntax}
\usepackage{xstring}
\usepackage{minted}
\usepackage{lmodern}
\usepackage{multicol}
\usepackage{xpatch}
\usepackage{booktabs}
\usepackage{multicol}
\usepackage{hyperref}
\usepackage{cleveref}

%% Hack to get rid of boxes around invalid chars in minted formatted text
%% https://tex.stackexchange.com/questions/45756/inline-code-and-short-verb-with-minted/148479
\makeatletter
\AtBeginEnvironment{minted}{\dontdofcolorbox}
\def\dontdofcolorbox{\renewcommand\fcolorbox[4][]{##4}}
\makeatother
\xpatchcmd{\mintinline}{\minted@fvset}{\minted@fvset\dontdofcolorbox}{}{}

\newminted{json}{fontsize=\small}
\newminted{c}{fontsize=\small}
\newmintinline{haskell}{}
\newmintinline{json}{}

\title{SME intermediate language specification}
\author{Truls Asheim {\tt <truls@asheim.dk>}}
\date{}

\begin{document}
\maketitle

\section{Introduction}
This document describes the intermediate representation (or language) for
representing SME networks. The concrete syntax for the language is described in
BNF format alongside the JSON serialization of the corresponding AST. The JSON
schema used is automatically derived from the Haskell definition of the SMEIL
(given in full in \Cref{hsast}) by the \texttt{Text.Aeson} library. The names
used in the datatypes are transformed from upper and lower camel case to
a hyphen-separated lower case sequence.

\subsection{Design Principles}
\begin{description}
\item[Language independence.] Since SME networks can be written in several
  different languages, the IR should not contain any required elements which are
  specific to one particular language.
\item[Structural richness.] A primary goal of the SME model is that the
  generated VHDL code should be readable and have a relation with the original
  source code. Therefore, contrary to the goal of most intermediate languages,
  SMEIL should have rich constructs for specifying the structure of SME
  networks.
\item[Readability.] Ensuring that the IR has a readable and accessible
  representation aids debugging and makes it possible to understand SME programs
  stored SMEIL. For this reason, SMEIL also have a human-readable concrete
  syntax.
\item[Composibility.] The IR should provide unrestricted composability to ensure
  that networks can be subdivided for optimal flexibility.
\end{description}

\subsection{Notation}
\paragraph{JSON} The JSON serialization format is specified as using fragments of standard
JSON-notation with a couple of additions.
\begin{itemize}
\item Names enclosed in angle-brackets should be replaced by the JSON fragment
  that their name refers to. References to terminals should be replaced by their
  definition given in BNF notation.
\item The pipe symbol, \verb!|!, is used to indicate choice in cases where an
  object can contain only one of several possible keys. For example, the fragment
\begin{jsoncode}  
[ { "element": { "network": <network>
               | "bus": <bus> } }
]
\end{jsoncode}
  describes a list of objects containing \textit{either} a single key named
  \texttt{network} followed by a network definition or a single key named
  \texttt{bus} followed by a bus definition. This is the representation of
  sum-types the AST definitions.
\item Values enclosed in \verb!?! characters are optional. \texttt{null}
  indicates that the value is absent. For example, the fragment
  \begin{jsoncode}
{ "count": ?<expr>?
, "size": <expr>
}
\end{jsoncode}
  describes an object where the \texttt{"count"} field can be either an expression or
  \texttt{null} while the \texttt{"size"} field is always an expression. This is
  the representation of a \haskellinline{Maybe a} value in the SME definition.
\item Square brackets, e.g. \verb![ el ]!, denotes a list containing zero or
  more objects of type \texttt{el}. While empty lists will always be
  successfully deserialized where lists are allowed, the resulting SMEIL AST may
  not represent valid SMEIL. The BNF description of the corresponding SMEIL
  construct specifies when an empty list is allowed and when an object must
  occur at least once. The exception to this is lists containing exactly two
  elements. These corresponds to tuples.
\item Parentheses are used for grouping multiple choices which are not JSON
  object keys (wrapped in curly-braces). For example, the type of binary
  operator in an expression is described as a choice between strings:
\begin{jsoncode}
{ "bin-op": ( "plus-op"
            | "minus-op"
            | ...
            )
\end{jsoncode}

\end{itemize}
\paragraph{BNF} The notation used mostly follows conventions of standard
(non-extended) BNF.

% \paragraph{Components}
% \begin{description}
%   %\item
% \end{description}



\section{Language Constructs}

%\setlength{\grammarparsep}{1pt plus 1pt minus 1pt} % increase separation between rules
\setlength{\grammarindent}{10em} % increase separation between LHS/RHS

\subsection{Top level constructs}

The fundamental building blocks of SME models are networks and processes. A
process is a concurrently run block of sequential code while networks describes
the invocations of

\paragraph{Grammar}

\begin{grammar}
  <design-file> ::= <design-unit> \{ <design-unit> \}

  <design-unit> ::= \{ <import-stm> \} <unit-element> \\ \{ <unit-element> \}

  <unit-element> ::= <network>
  \alt <process>
\end{grammar}

\paragraph{JSON}

\begin{jsoncode}
{ "design-file": [
  {"design-unit":
    [ { "imports": [<import>]
      , "unit-element": [<network>]
      }
    ]
  }
  ]
}

{ "unit-proc": { "processes":  [ <process> ] }
| "unit-net": { "networks": [ <network> ] }
}
\end{jsoncode}

\subsection{Network structure}

\subsubsection{Elements}
\paragraph{BNF}
\begin{grammar}
  <import> ::= `import' <name> `;'

  <network> ::= `network' <ident> `(' [ <params> ] `)' `{' <network-decl> `}'

  <network-decl> ::= <instance>
  \alt <bus-decl>
  \alt <const-decl>

  %<process> ::= [ `sync' | `async' ] [ `simulation' ] `proc' <ident> \\ `(' [
  <process> ::= [ `sync' | `async' ] `proc' <ident> \\ `(' [
  <params> ] `)' \{ declaration \} `{' \{ <statement> \} `}'
\end{grammar}

\paragraph{JSON}
\begin{jsoncode}
 {"import": <ident>}

 {"network": { "name": <ident>
             , "params": [ <param> ]
             , "net-decls": [ <network-decl> ]
             }
 }

// network-decl
{ "net-inst": { "inst": <instance> }
| "net-bus": { "bus": <bus-decl> }
| "net-const": { "const": <const-decl> }
}

 {"process": { "name": <ident>
             , "params": [ <param> ]
             , "decls": [ <declaration> ]
             , "body": [ <statement> ]
             , "clocked": bool
             }
}
\end{jsoncode}

\subsubsection{Definitions}
\paragraph{BNF}
\begin{grammar}

  <instance> ::= `instance' <instance-name> `of' <ident> \\`(' [ <param-map> { `,' <param-map> } ]`)' `;'

  <instance-name> ::= <ident> (named instance)
  \alt `_' (anonymous instance)

  <param-map> ::= [ <ident> `:' ] <expression>

  <params> ::= <param> \{ , <param> \}

  <param> ::= <direction> <ident>

  <enum> ::= `enum' <ident> `{' <enum-field> \{ `,' <enum-field>  \} `}' `;'

  <enum-field> ::= <ident> [ `=' <expr> ]

  <bus-decl> ::= [ `exposed' ] `bus' <ident> `\{' <bus-signal-decls> `\}'  `;'

  <bus-signal-decls> ::= <bus-signal-decl> \{ [ `,' <bus-signal-decl> ] \}

  <bus-signal-decl> ::= <ident> `:' <type> [ `=' <expression> ] [ <range> ] `;'

  <var-decl> ::= `var' <ident> `:' \\ <type-name> [ `=' <expression> ] [ <range> ] `;'

  <range> ::= `range' <expression> `to' <expression>

  <const-decl> ::= `const' <ident> `:' <type-name> [ `=' <expression> ] `;'

  <function> ::= `func' <ident> `{' { <statement> } `}'

  <declaration> ::= <var-decl>
  \alt <const-decl>
  \alt <bus-decl>
  \alt <enum>
  \alt <function>

\end{grammar}

\paragraph{JSON}
\begin{jsoncode}
{"instance": { "inst-name": ?<ident>?
             , "el-name": <ident>
             , "params": [ [ ?"ident"?, <expr> ] ]
             }
}

{"enumeration": { "name": <ident>
                , "fields": [ [<ident>, ?<expr>?] ]
                }
}

{"range": { "from": <expr>
          , "to": <expr>
          }
}

{"bus": { "name": <ident>
        , "exposed" <bool>
        , "signals": [ <bus-signal> ]
        }
}


{"bus-signal": { "name": <ident>
               , "ty": <type>
               , "value": ?<expr>?
               , "range": ?<range>?
               }
}

{"variable": { "name": <ident>
             , "ty": <type>
             , "val": ?<expr>?
             , "range", ?<range>?
             }
}

{"constant": { "name": <ident>
             , "ty": <type>
             , "val": <expr>
             }
}

{"function": { "name": <ident>
             , "params": [ <ident> ]
             , "body": [ <statement> ]
             }
}

{"declaration": { "var-decl": <variable>
                | "const-decl": <constant>
                | "bus-decl": <bus>
                }
}
\end{jsoncode}

\subsection{Statements}

\paragraph{BNF}

\begin{grammar}

  <statement> ::= <name> `=' <expression> `;'
  \alt `if' `(' <condition> `)' `{' \{ <statement> \} `}' \\ \{ <elif-block>
    `}' `[' <else-block> `]'
  \alt `for' <ident> `=' <expression> `to' <expression> \\ `{' \{ <statement> \} `}'
  \alt `switch' <expression> `where' \\ `{' <switch-case> \{ <switch-case> \} `}'
    [ `default` `:' <statement> \{ <statement> \} ]
  \alt `barrier' `;'
  \alt `break' `;'
  \alt `return' [ <expr> ] `;'

  <switch-case> ::= `case' <expression> `:' \{ <statement> \}

  <elif-block> ::= `elif `(' <condition> `)' `{' \{ <statement> \} `}'

  <else-block> ::= `else' `{' \{ <statement> \} `}'

\end{grammar}

\paragraph{JSON}
\begin{jsoncode}
{ "assign" : { "var": <name>
             , "val" <expr>
             }
| "if": { "cond": <expr>
        , "body": [ <statement> ]
        , "elif": [ [ <expr>, [ <statement> ] ] ]
        , "els":  [ <expr>, [ <statement> ] ]
        }
| "for": { "var": <ident>
         , "from": <expr>
         , "to": <expr>
         , "body": [ <statement> ]
         }
| "switch": { "value": <expr>
            , "cases": [ [ <expr>, [ <statement> ] ] ]
            , "default-case": [ <statement> ]
            }
| "barrier": []
| "break": []
| "return": { "val": <expr> }
}

\end{jsoncode}
\subsection{Expressions}
\paragraph{BNF}
\begin{grammar}
  <expression> ::= <name>
  \alt <literal>
  \alt <expression> <bin-op> <expression>
  \alt <un-op> <expression>
  \alt <name> `(' \{ <expression> \}  `)' (function call)
  \alt `(' <expression> `)'

  <bin-op> ::= `+' (addition)
  \alt `-' (subtraction)
  \alt `*' (multiplication)
  \alt `/' (division)
  \alt `\%' (modulo)
  \alt `==' (equal)
  \alt `!=' (not equal)
  \alt `\verb!<<!' (shift left)
  \alt `\verb!>>!' (shift right)
  \alt `<' (less than)
  \alt `>' (greater than)
  \alt `<=' (greater than or equal)
  \alt `>=' (less than or equal)
  \alt `\&' (bitwise-and)
  \alt `|' (bitwise-or)
  \alt `^' (bitwise-xor)

  <un-op> ::= `-' (negation)
  \alt `+' (identity)
  \alt `!' (bitwise-not)

\end{grammar}
\paragraph{JSON}
\begin{jsoncode}
{ "binary": { "bin-op": <bin-op>
            , "left": <expr>
            , "right": <expr>
            }
| "unary": { "un-op": <un-op>
             "expr": <expr>
           }
| "prim-lit": { "lit": <literal> }
| "prim-name": { "name": <name> }
| "fun-call": { "name": <name>
              , "params": [ <expr> ]
              }
}

{ "bin-op": ( "plus-op"
            | "minus-op"
            | "mul-op"
            | "div-op"
            | "mod-op"
            | "eq-op"
            | "neq-op"
            | "sll-op"
            | "srl-op"
            | "lt-op"
            | "gt-op"
            | "leq-op"
            | "geq-op"
            | "and-op"
            | "or-op"
            | "xor-op"
            )
}

{ "un-op": ( "un-plus"
           | "un-minus"
           | "not-op"
           )
}
\end{jsoncode}

\subsubsection{Operator precedence}

The operator precedence of SMEIL is listed \Cref{tab:ops}

\begin{table}[H]
  \centering
\begin{tabular}{cc}
  \toprule
  \textbf{Precedence} & \textbf{Operators}\\
  \midrule
  0 & \verb!+! \verb!-! \verb|!| (unary)\\
  1 & * / \% \\
  2 & + - \\
  3 & \verb!<<! \verb!>>! \\
  4 & < > <= >= \\
  5 & \verb!&! \verb!^! \verb!|! \\
  \bottomrule
\end{tabular}
\caption{Operator precedence of SMEIL}
\label{tab:ops}
\end{table}

\subsection{Lexical elements}
\paragraph{BNF}
\begin{grammar}

  <direction> ::= `in' (input signal)
  \alt `out' (output signal)
  \alt `const' (constant input value)

  <type> ::= `i' <integer> (signed integer)
  \alt `u' <integer> (unsigned integer)
  \alt `f32' (single-precision floating point)
  \alt `f64' (double-precision floating point)
  \alt `bool' (boolean value)
  \alt `[' [ <expression> ] `]' <type> (array of type)


  <literal> ::= <integer>
  \alt <floating>
  \alt `"'\{ <char> \}`"' (string literal)
  \alt `[' <expression> \{ `,' <expression> \} `]' (array literal)
  \alt `true'
  \alt `false'

  <ident> ::= <letter> \{ ( <letter> | <num> | `_' | `-' ) \} (identifier)

  % TODO: Make more specific
  <name> ::= <ident>
  \alt <name> `.' <ident> (hierarchical accessor)
  \alt <name> `[' <expression> `]' (array element access)

  <integer> ::= <number> { <number> } (decimal number)
  \alt `0x' <hex-digit> { <hex-digit> } (hexadecimal number)
  \alt `0o' <octal-digit> { <hex-digit> } (octal number)

  <alpha-num> ::= <alpha>

\end{grammar}

\paragraph{JSON}
\begin{jsoncode}
{"direction": ( "in"
              | "out"
              | "const" )
}

{ "signed": { "size": <int> }
| "unsigned": { "size": <int> }
| "single": []
| "double": []
| "bool": []
| "array": { "arr-length": ?<expression>?
           , "inner-ty": <type>
           }
}

{ "name": { "ident": <ident> }
| "hier-access": { "idents": [ <ident> ] }
| "array-access": { "name": <name>
                  , "index": <expression>
                  }
}

{ "lit-int": <int>
| "lit-string": <string>
| "lit-array": [ <expr> ]
| "lit-true": []
}
\end{jsoncode}

\subsection{Keywords}
\begin{multicols}{3}
  \begin{itemize}
  \item async
  \item bus
  \item const
  \item elif
  \item else
  \item exposed
  \item for
  \item func
  \item if
  \item import
  \item in
  \item instance
  \item network
  \item of
  \item out
  \item proc
  \item range
  \item simulation
  \item switch
  \item sync
  \item to
  \item var
  \item where
  \end{itemize}
\end{multicols}

\section{Example programs}

\subsection{allops.sme}
\inputminted[fontsize=\small]{c}{samples/allops3.sme}

\subsection{simplefifo.sme}
\inputminted[fontsize=\small]{c}{samples/simplefifo.sme}

\newpage
\appendix
\section{Haskell Implementation of the SMEIL AST}
\label{hsast}
\inputminted[fontsize=\small]{haskell}{../../src/Language/SMEIL/Syntax.hs}
\end{document}

%%% Local Variables:
%%% mode: latex
%%% TeX-master: t
%%% TeX-command-extra-options: "-enable-write18"
%%% End:
